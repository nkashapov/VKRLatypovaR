\anonsection{Введение}

\par Различие культур обуславливают языковые различия, в особенности они ярко выражаются   в лексике, так как номинативные средства языка наиболее прямо связаны с внеязыковой действительностью. В любом языке и диалекте есть слова, которые нельзя перевести на другой язык одним словом. Многие исследователи, такие как  А.Д. Швейцер,  Я.И. Рецкер, Л.С. Бархударов, В.Н. Комиссаров,  занимались проблемой безэквивалентной лексики. Проблема языкового барьера с давних времен ограничивало международное общение. Этим может быть объяснена тенденция мирового сообщества к созданию универсального международного языка. Сегодня статус универсального международного языка принадлежит английскому языку.Однако он, как и все естественные языки, обладает безэквивалентной лексикой, что затрудняет его понимание и перевод на другие языковые системы.

\par \textbf{Актуальность исследования} обусловлена: с одной стороны, важности  проблемы соотношения  культуры и языка; с другой стороны интересом теоретиков перевода к проблеме передачи безэквивалентной лексики и многочисленными ошибками, допускаемыми при ее передаче на другой язык.

\par \textbf{Целью} нашего исследования является выявление особенностей перевода безэквивалентной лексики с английского языка на русский и её типологизация на материале песенных текстов группы Queen. Реализация цели предполагает последовательное решение следующих \textbf{задач}:
\begin{enumerate}
    \item проанализировать разработанность проблемы перевода  единиц безэквивалентной лексики в отечественной и зарубежной лингвистике;
    \item  выявить способы  перевода единиц безэквивалентной лексики (БЭЛ) c одного языка на другой;
    \item проанализировать соответствия перевода единиц БЭЛ, представленных на сайтах amalgama-lab.com, lyrsence.com, mirpesen. com, оригиналу на  материале песен группы Queen;
\item произвести лексико-семантический анализ единиц БЭЛ выявленных в песенных текстах группы Queen.
\end{enumerate} 

\par \textbf{Объектом} исследования являются единицы безэквивалентной лексики в английском  языке.
\par \textbf{Предметом} исследования являются переводческие трансформации при переводе фразеологических единиц с английского языка на русский язык.  
\par \textbf{Языковым материалом} послужили песенные тексты группы «Queen».
\par \textbf{К основным методам} исследования относятся:
\begin{enumerate}[label=\arabic*)] 
    \item Общенаучные (анализ, синтез, умозаключение, категоризация). С их помощью происходит анализ разработанности проблемы и выявление неисследованных областей, формулирование выводов и результатов исследования;
    \item Частнолингвистические. Метод сплошной выборки на материале песенных текстов; метод сопоставительного анализа разноструктурных языков (английского и русского); метод фразеологической идентификации.
\end{enumerate}
\par \textbf{Научная новизна} работы состоит в попытке сопоставительного анализа оригинального и переводного текстов и последующей типологизацией переводческих трансформаций  использованных при переводе  с английского языка на русский.
\par \textbf{Теоретическая значимость} усматривается в систематизации и углублении сведений по безквивалентности, переводу безэквивалентных единиц и классификации переводческих трансформаций использованных при переводе с английского языка на русский.
\par \textbf{Практическая ценность} определяется возможностью использования результатов исследования на уроках английского языка при изучении безэквивалентных лексических единиц, курсах по фразеологии, лексикологии, практических курсах английского языка и перевода.
\par Работа прошла \textbf{апробацию} в Муниципальном Бюджетном общеобразовательном  учреждении "Уруссинская средняя общеобразовательная школа №3" Ютазинского муниципального  района Республики Татарстан во время урока английского языка в 10 классе, также во время  итоговой научно-практической конференции студентов КФУ, где автор представил материалы исследования в форме доклада.
\par \textbf{Структура работы.} Работа состоит из введения, обосновывающего актуальность работы, его цели, задачи и пр.; теоретической главы, включающей обзор литературы по теме исследования; практической главы с результатами научного исследования проблемы, выводов, списка использованной литературы и заключения. (К работе также прилагается… (перечисляется состав приложения).  



\clearpage

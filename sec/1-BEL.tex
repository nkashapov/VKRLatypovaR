\section{Перевод безэквивалентной лексики}

\par В теории перевода, под переводом подразумевают передачу coдepжaния ycтнoгo выскaзывaния или пиcьмeннoгo тeкcтa cpeдcтвaми дpyгoгo языкa.  Ирина Сергеевна Алексеева под переводом понимает деятельность переводчика, заключаюущюся в различном перевыражении текста одного языка, в текст другого. (Введение в переводоведение.- 2004) 
\par Венедикт Степанович Виноградов более точно определяет перевод так как понимает под ним не только деятельность переводчика, но и результат его работы. По В.С. Виноградову «Перевод - это вызванный общественной необходимостью процесс и результат передачи информации, выраженных в письменном или устном тексте на одном языке, посредством эквивалентного текста на другом языке». [Введение в переводоведение, 2001]
В теории перевода различают понятия эквивалентности и адекватности перевода. Вопрос определения дефиниций данных понятий затруднен спорами исследователей, одни считают их синонимичными, другие, хоть и признают их взаимосвязанность, оспаривают их полное тождество. Например, для В.Н. Комиссарова понятие адекватности перевода является более широким, нежели эквивалентность. Под адекватностью перевода он понимает выражение необходимого объема и полноты исходного текста в определенной ситуации. Он приравнивает адекватный перевод  хорошему и достаточному переводу, так как полное соответствие переведенного текста оригиналу невозможно.   Эквивалентность исследователи рассматривают  как смысловую общность языковых и речевых единиц, отождествляемых друг с другом. [Комиссаров 2002: 116-117]. 

\par Под безэквивалентной лексикой стоит понимать, единицы лексики одного языка полностью или частично отсутствующих в лексической системе другого языка. Е.М. Верещагин и В,Г.Костомаров предложили термин безэквивалентной лексики, рассматривая его как «слова, служащие для выражения понятий, отсутствующих в иной культуре и ином языке, слова, относящиеся к частным культурным элементам». [Верещагин, Костомаров, 1983, с. 53] Тем не менее терминология может отличаться в разных исследовательских и переводческих кругах, таким образом, можно встретить труды, в которых безэквивалентная лексика заменена другими  терминами, например, варваризмы, реалии, локализмы, экзотизмы, этнографизмы и некоторые другие. Более того разные термины подкрепляются соответствующими определениями и несут разные значения, отмечая одни признаки лексических единиц и опуская другие. Для того, чтобы дальнейшее повествование не вызывало путаницы следует определиться с используемой терминологией в данной работе и указать на оттенки значений представленных терминов. Несмотря на большое количество синонимов БЭЛ наиболее близким стал термин  реалия. БЭЛ в сущности является  термином рассчитанным для переводческих наук, сопоставительного языкознания и контрастивной лингвистики, так как он предназначен для наук  ведущим методом исследования которых является сравнение категорий одного языка с категориями другого. Поэтому  определение безэквивалентности опирается на принятое в теории перевода представление об эквиваленте, а именно о постоянном равнозначном соответствии между единицами исходного и переводного текстов, которое не зависит от контекста. Таким образом, безэквивалентной лексикой будут считаться лексические и фразеологические единицы, не имеющие, постоянных, не зависящих от контекста, эквивалентов в переводящем языке. С. Влахов и С.Флорин считают, что  реалия остается реалией безотносительно к тому или иному языку, в то время как безэквивалентность устанавливается в рамках пары языков.
